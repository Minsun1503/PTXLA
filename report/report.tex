\documentclass[12pt, a4paper]{article}
\usepackage[utf8]{vietnam} % Hỗ trợ tiếng Việt
\usepackage{geometry}      % Căn lề
\usepackage{graphicx}      % Chèn ảnh
\usepackage{hyperref}      % Tạo link liên kết
\usepackage{listings}      % Hiển thị đoạn code
\usepackage{xcolor}        % Màu sắc
\usepackage{float}

% Cấu hình lề trang
\geometry{
	a4paper,
	total={170mm,257mm},
	left=25mm,
	top=25mm,
	right=25mm,
	bottom=25mm
}

% Cấu hình hiển thị code (Python/Bash)
\definecolor{codegreen}{rgb}{0,0.6,0}
\definecolor{codegray}{rgb}{0.5,0.5,0.5}
\definecolor{codepurple}{rgb}{0.58,0,0.82}
\definecolor{backcolour}{rgb}{0.95,0.95,0.92}

\lstdefinestyle{mystyle}{
	backgroundcolor=\color{backcolour},   
	commentstyle=\color{codegreen},
	keywordstyle=\color{magenta},
	numberstyle=\tiny\color{codegray},
	stringstyle=\color{codepurple},
	basicstyle=\ttfamily\small,
	breakatwhitespace=false,         
	breaklines=true,                 
	captionpos=b,                    
	keepspaces=true,                 
	numbers=left,                    
	numbersep=5pt,                  
	showspaces=false,                
	showstringspaces=false,
	showtabs=false,                  
	tabsize=2,
	frame=single % Khung viền quanh code
}

\lstset{style=mystyle}

% --- THÔNG TIN ĐẦU FILE ---
\title{
	\textbf{\Large ĐỒ ÁN XỬ LÝ ẢNH: HỆ THỐNG CHẤM ĐIỂM TRẮC NGHIỆM VÀ NHẬN DẠNG VĂN BẢN TỰ ĐỘNG (OMR \& OCR)}
}
\author{Nhóm thực hiện: Nguyễn Thế Minh Nhật}
\date{\today}

\begin{document}
	
	\maketitle
	\tableofcontents
	\newpage
	
	% --- NỘI DUNG CHÍNH ---
	
	\section{Giới thiệu chung}
	Trong bối cảnh chuyển đổi số giáo dục, việc tự động hóa các quy trình thủ công như chấm thi và quản lý thông tin thí sinh ngày càng trở nên cấp thiết. Đồ án này trình bày giải pháp xây dựng một hệ thống xử lý ảnh thông minh, tích hợp hai chức năng chính:
	\begin{enumerate}
		\item \textbf{Chấm điểm trắc nghiệm tự động (Optical Mark Recognition - OMR):} Nhận dạng và chấm điểm các câu trả lời trên phiếu trắc nghiệm.
		\item \textbf{Nhận dạng văn bản tự động (Optical Character Recognition - OCR):} Tự động phát hiện và trích xuất thông tin văn bản (như Tên, MSSV, Mã đề) từ các vùng bên ngoài phiếu trả lời.
	\end{enumerate}
	Hệ thống được thiết kế để tối ưu hóa hiệu suất, giảm thiểu sai sót của con người và cung cấp kết quả một cách nhanh chóng, trực quan.
	
	\subsection{Các tính năng chính}
	\begin{itemize}
		\item \textbf{Xử lý từ file PDF:} Hệ thống có khả năng đọc và xử lý trực tiếp file PDF chứa ảnh scan của phiếu làm bài.
		\item \textbf{Tự động định vị và căn chỉnh phiếu:} Tự động tìm kiếm khung viền của phiếu trong ảnh và áp dụng biến đổi hình học để làm phẳng, loại bỏ biến dạng phối cảnh.
		\item \textbf{Thiết lập bán tự động thông minh:}
		    \begin{itemize}
		        \item \textit{OMR:} Cung cấp công cụ tương tác để người dùng chỉ cần chọn 2 điểm neo (góc trên-trái và dưới-phải) cho khối trắc nghiệm trong lần đầu tiên.
		        \item \textit{OCR:} Hoàn toàn tự động phát hiện các vùng chứa văn bản (tên, mã số...) nằm bên ngoài khối trắc nghiệm mà không cần sự can thiệp của người dùng.
		    \end{itemize}
		\item \textbf{Chấm điểm OMR chính xác:} Chấm điểm và so sánh với đáp án, đưa ra điểm số cuối cùng.
		\item \textbf{Trích xuất thông tin OCR:} Sử dụng model EasyOCR để nhận dạng và trích xuất nội dung văn bản từ các vùng đã được tự động phát hiện.
		\item \textbf{Trực quan hóa và lưu trữ kết quả:} Hiển thị kết quả chấm (đúng/sai) trực tiếp trên ảnh, tạo ảnh báo cáo điểm và lưu toàn bộ kết quả (ảnh, dữ liệu OCR) vào thư mục đầu ra.
	\end{itemize}
    
    \subsection{Các công nghệ sử dụng}
    \begin{itemize}
        \item \textbf{Python 3:} Ngôn ngữ lập trình chính của dự án.
        \item \textbf{OpenCV (Open Source Computer Vision Library):} Thư viện mã nguồn mở hàng đầu cho các tác vụ xử lý ảnh, được sử dụng cho hầu hết các công đoạn từ đọc ảnh, biến đổi hình học, đến phát hiện đối tượng.
        \item \textbf{NumPy:} Thư viện nền tảng cho tính toán khoa học trong Python, cung cấp cấu trúc mảng đa chiều hiệu suất cao để biểu diễn và tính toán ma trận điểm ảnh.
        \item \textbf{EasyOCR:} Một thư viện OCR mạnh mẽ và dễ sử dụng, hỗ trợ nhiều ngôn ngữ (bao gồm tiếng Việt), được dùng để trích xuất văn bản.
        \item \textbf{pdf2image:} Thư viện tiện ích giúp chuyển đổi các trang của file PDF thành đối tượng ảnh mà OpenCV có thể xử lý. Nó hoạt động như một lớp vỏ (wrapper) cho tiện ích Poppler.
        \item \textbf{Poppler:} Một thư viện render PDF, là phụ thuộc hệ thống bắt buộc của \texttt{pdf2image}.
    \end{itemize}
	
	\section{Cấu trúc dự án}
	Cây thư mục của dự án được tổ chức một cách khoa học để dễ dàng quản lý và mở rộng:
	
	\begin{description}
		\item[\texttt{main.py}:] File thực thi chính, điều phối toàn bộ luồng hoạt động của hệ thống.
		\item[\texttt{requirements.txt}:] Danh sách các thư viện Python cần thiết.
		\item[\texttt{src/}:] Thư mục chứa mã nguồn xử lý lõi.
		\begin{itemize}
			\item \texttt{config.py}: File cấu hình tập trung, chứa tất cả các đường dẫn, ngưỡng và tham số quan trọng.
			\item \texttt{pre\_processing.py}: Các hàm tiền xử lý ảnh (chuyển đổi PDF, tìm và bẻ thẳng phiếu).
			\item \texttt{omr\_logic.py}: Logic chấm điểm trắc nghiệm (tạo lưới tọa độ, phân tích ô được tô).
			\item \texttt{ocr\_logic.py}: Logic trích xuất văn bản (sử dụng EasyOCR trên các vùng ảnh đầu vào).
			\item \texttt{measure\_tool.py}: Công cụ tương tác cho việc thiết lập tọa độ lần đầu.
            \item \texttt{utils.py}: Các hàm tiện ích chung (sắp xếp điểm, biến đổi phối cảnh, xử lý contour).
            \item \texttt{ui.py}: Các hàm quản lý giao diện người dùng (hộp thoại, hiển thị kết quả).
		\end{itemize}
		\item[\texttt{data/}:] Chứa dữ liệu đầu vào.
		    \begin{itemize}
		        \item \texttt{raw/}: Chứa file PDF phiếu làm bài.
		        \item \texttt{answer/}: Chứa file đáp án cho phần OMR.
		        \item \texttt{template/}: Chứa file tọa độ được lưu sau lần thiết lập đầu tiên.
		    \end{itemize}
		\item[\texttt{output/}:] Thư mục lưu trữ tất cả các kết quả đầu ra (ảnh, file JSON).
		\item[\texttt{poppler-25.11.0/}:] Thư viện Poppler cần thiết cho việc xử lý PDF trên Windows.
	\end{description}
	
	\section{Cài đặt và Hướng dẫn sử dụng}
	
	
	
	\subsection{Yêu cầu về môi trường}
	
	\begin{enumerate}
	
		\item \textbf{Python:} Phiên bản 3.8 trở lên.
	
		\item \textbf{Poppler:} Do hệ thống cần xử lý file PDF, Poppler là một yêu cầu bắt buộc.
	
		      \begin{itemize}
	
			      \item Dự án đã đi kèm thư mục \texttt{poppler-25.11.0} cho Windows.
	
			      \item \textbf{Quan trọng:} Người dùng cần thêm đường dẫn đầy đủ tới thư mục \texttt{poppler-25.11.0/Library/bin} vào biến môi trường \texttt{Path} của hệ thống. Nếu không có bước này, chương trình sẽ báo lỗi không tìm thấy Poppler.
	
		      \end{itemize}
	
	\end{enumerate}
	
	
	
	\subsection{Cài đặt các thư viện Python}
	
	Mở một cửa sổ dòng lệnh (terminal) trong thư mục gốc của dự án và chạy lệnh sau để cài đặt tất cả các thư viện cần thiết:
	
	\begin{lstlisting}[language=bash]
	
	pip install -r requirements.txt
	
	\end{lstlisting}
	
	\textit{Lưu ý: Trong lần chạy đầu tiên, thư viện \texttt{easyocr} có thể sẽ tự động tải về các model nhận dạng. Đây là quá trình chỉ diễn ra một lần và cần có kết nối Internet.}
	
	
	
	\subsection{Cấu hình hệ thống}
	
	Tất cả các cấu hình quan trọng đều được tập trung tại file \texttt{src/config.py}.
	
	\begin{itemize}
	
		\item \texttt{PDF\_PATH}: Đường dẫn tới file PDF phiếu làm bài đầu vào.
	
		\item \texttt{ANSWER\_KEY\_PATH}: Đường dẫn tới file \texttt{.csv} chứa đáp án của phần trắc nghiệm.
	
		\item \texttt{COORDINATES\_PATH}: Đường dẫn để lưu file template tọa độ sau lần thiết lập đầu tiên.
	
	\end{itemize}
	
	Người dùng cần đảm bảo các file dữ liệu (phiếu làm bài, đáp án) được đặt đúng vị trí và các đường dẫn trong file config là chính xác.
	
	
	
	\subsection{Hướng dẫn vận hành}
	
	\begin{enumerate}
	
		\item \textbf{Chạy chương trình:}
	
		      \begin{lstlisting}[language=bash]
	
	python main.py
	
	\end{lstlisting}
	
		\item \textbf{Thiết lập lần đầu (First-Time Setup):}
	
		      Nếu hệ thống không tìm thấy file tọa độ template, một hộp thoại sẽ hiện lên hỏi quyền được bắt đầu quá trình thiết lập.
	
		      \begin{itemize}
	
			      \item \textbf{Giai đoạn 1 (Thủ công):} Một cửa sổ hiển thị ảnh phiếu sẽ mở ra. Người dùng cần click chuột vào 2 vị trí theo hướng dẫn trên console: tâm của ô đáp án \textbf{trên cùng bên trái} và tâm của ô \textbf{dưới cùng bên phải} của khối trắc nghiệm.
	
			      \item \textbf{Giai đoạn 2 (Tự động):} Ngay sau khi 2 điểm được chọn, cửa sổ sẽ đóng lại. Hệ thống sẽ tự động chạy nền để tìm các vùng văn bản OCR.
	
			      \item Toàn bộ tọa độ sau đó được lưu vào file \texttt{coordinates.json}.
	
		      \end{itemize}
	
		\item \textbf{Các lần chạy sau:}
	
		      Chương trình sẽ tự động đọc file \texttt{coordinates.json} đã lưu và thực hiện toàn bộ quy trình chấm điểm, nhận dạng mà không cần bất kỳ tương tác nào.
	
	\end{enumerate}
	
	
	
	
	
	\section{Kết quả thực nghiệm và Đánh giá}

\subsection{Phương pháp thử nghiệm}
Do hệ thống không trải qua giai đoạn huấn luyện (OMR là thuật toán kinh điển, OCR sử dụng model có sẵn), khái niệm tập dữ liệu "train/test" không được áp dụng. Thay vào đó, hệ thống được thử nghiệm dưới dạng một nghiên cứu tình huống (case study) với một bộ dữ liệu đơn lẻ:
\begin{itemize}
    \item \textbf{Dữ liệu đầu vào:} Một file PDF mẫu (\texttt{Mau\_de\_thi\_co\_dap\_an.pdf}) chứa ảnh scan của một phiếu trả lời đã được điền.
    \item \textbf{Đáp án:} Một file CSV (\texttt{answer\_key.csv}) chứa danh sách các đáp án đúng.
\end{itemize}
Mục tiêu của thử nghiệm là để xác minh tính năng (proof-of-concept) và quan sát hoạt động của hệ thống trên một ví dụ cụ thể, chứ không phải để đo lường hiệu suất trên một tập dữ liệu lớn.

\subsection{Kết quả quan sát được}
Khi thực thi với dữ liệu mẫu, hệ thống cho ra các kết quả sau:
\begin{itemize}
    \item \textbf{Kết quả OMR:} Chấm điểm thành công và tạo ra ảnh \texttt{scoring\_result.png} (trực quan hóa đáp án đúng/sai) và ảnh \texttt{score.png} (báo cáo điểm số). Cụ thể, hệ thống ghi nhận \textbf{4/20} câu đúng, tương đương \textbf{2.0/10 điểm}.
    \item \textbf{Kết quả OCR:} Trích xuất thành công chuỗi \texttt{'2111Ũ355 153'} từ vùng thông tin thí sinh và lưu vào file \texttt{ocr\_results.json}.
\end{itemize}
\textit{Các hình ảnh kết quả được tạo ra trong thư mục \texttt{output/} có thể được chèn vào đây để minh họa cho hoạt động của chương trình.}

\subsection{Đánh giá và Nhìn nhận các hạn chế}
Phần đánh giá này nhìn nhận lại dự án dựa trên các tiêu chí khoa học và phản hồi thực tế, thay vì chỉ đánh giá trên bề mặt.

\subsubsection{Về "Cấu trúc mô hình"}
Thuật ngữ "mô hình" cần được hiểu đúng trong bối cảnh của dự án:
\begin{itemize}
    \item \textbf{OMR:} Không phải là một mô hình học máy. Đây là một \textbf{thuật toán} xử lý ảnh dựa trên các quy tắc được định sẵn (ngưỡng và đếm pixel). Nó có cấu trúc đơn giản, hiệu quả cho mẫu phiếu cụ thể nhưng không có khả năng "học" hay tổng quát hóa.
    \item \textbf{OCR:} Hệ thống chỉ \textbf{sử dụng} một model đã được huấn luyện trước (\texttt{EasyOCR}). Đồ án không thực hiện việc xây dựng, huấn luyện hay tinh chỉnh bất kỳ một mô hình học sâu nào. Do đó, phần cấu trúc model OCR thực chất là cấu trúc của thư viện \texttt{EasyOCR}, không phải là một phần do dự án phát triển.
\end{itemize}

\subsubsection{Về "Tập dữ liệu" và "Kết quả phân tích"}
Đây là hạn chế lớn nhất của đồ án ở thời điểm hiện tại:
\begin{itemize}
    \item \textbf{Thiếu tập dữ liệu chuẩn:} Dự án chỉ được thử nghiệm trên một file mẫu duy nhất. Không có một tập dữ liệu đa dạng (nhiều phiếu thi, nhiều điều kiện scan, nhiều cách tô khác nhau) để có thể đưa ra các kết quả số liệu có ý nghĩa thống kê (ví dụ: độ chính xác trung bình, độ lệch chuẩn, ma trận nhầm lẫn).
    \item \textbf{Thiếu phân tích số liệu:} Do không có tập dữ liệu kiểm thử, các kết quả thu được (ví dụ: điểm 2.0/10) chỉ là một quan sát đơn lẻ. Không có cơ sở để phân tích sâu hơn về hiệu suất, các trường hợp lỗi, hay so sánh với các phương pháp khác một cách định lượng. Các nhận xét về "độ chính xác gần như tuyệt đối" ở các phiên bản trước của báo cáo là chủ quan và không có cơ sở khoa học.
\end{itemize}
Tóm lại, hệ thống hiện tại là một sản phẩm chứng minh ý tưởng (proof-of-concept) hoạt động tốt, nhưng để trở thành một dự án khoa học hoàn chỉnh, việc xây dựng một bộ dữ liệu và thực hiện một quy trình đánh giá nghiêm ngặt là bước đi cần thiết tiếp theo.

\section{Kết luận và Hướng phát triển}
	
	
	
	\subsection{Kết luận}
	
	Đồ án đã xây dựng thành công một hệ thống hoàn chỉnh có khả năng tự động hóa hai tác vụ quan trọng trong chấm thi: chấm điểm trắc nghiệm (OMR) và trích xuất thông tin thí sinh (OCR). Bằng cách kết hợp các thuật toán xử lý ảnh kinh điển của OpenCV và sức mạnh của model học sâu từ EasyOCR, hệ thống đã giải quyết được bài toán đặt ra với các kết quả chính:
	
	\begin{itemize}
	
	    \item Tự động hóa toàn bộ quy trình từ đọc file PDF đến xuất kết quả.
	
	    \item Giảm thiểu tối đa sự can thiệp của người dùng nhờ cơ chế thiết lập template thông minh và tự động phát hiện vùng OCR.
	
	    \item Cung cấp kết quả đầu ra đa dạng, trực quan và dễ dàng lưu trữ, quản lý.
	
	\end{itemize}
	
	Dự án là một minh chứng rõ ràng cho tiềm năng của Xử lý ảnh trong việc giải quyết các bài toán thực tế trong lĩnh vực giáo dục và xa hơn nữa.
	
	
	
	\subsection{Hạn chế của hệ thống}
	
	\begin{itemize}
	
		\item \textbf{Phụ thuộc vào cấu trúc phiếu:} Logic sinh tọa độ OMR hiện tại được thiết kế cho một mẫu phiếu nhất định. Việc thay đổi sang một mẫu phiếu có bố cục hoàn toàn khác (ví dụ: 2 cột câu hỏi) sẽ cần can thiệp vào mã nguồn.
	
		\item \textbf{Chất lượng ảnh hưởng lớn:} Chất lượng của ảnh scan (độ phân giải, độ tương phản, ánh sáng) ảnh hưởng trực tiếp đến độ chính xác của cả OMR và OCR.
	
		\item \textbf{Giao diện người dùng:} Hệ thống vẫn hoạt động chủ yếu trên giao diện dòng lệnh và các cửa sổ của OpenCV, chưa có một giao diện đồ họa (GUI) hoàn chỉnh.
	
	\end{itemize}
	
	
	
	\subsection{Hướng phát triển}
	
	\begin{itemize}
	
		\item \textbf{Xây dựng giao diện đồ họa (GUI):} Phát triển một ứng dụng desktop (sử dụng PyQt, Tkinter) hoặc web (sử dụng Flask, Django) để người dùng có thể dễ dàng tải lên file, cấu hình và xem kết quả một cách thân thiện hơn.
	
		\item \textbf{Nâng cao khả năng thích ứng:} Phát triển các thuật toán thông minh hơn để hệ thống có thể tự động nhận dạng bố cục của các loại phiếu trắc nghiệm khác nhau mà không cần cấu hình lại.
	
		\item \textbf{Xử lý theo lô (Batch Processing):} Cho phép người dùng tải lên nhiều file cùng lúc và hệ thống tự động xử lý hàng loạt.
	
		\item \textbf{Cải thiện độ chính xác OCR:} Thử nghiệm tích hợp các engine OCR khác hoặc tinh chỉnh model để cải thiện độ chính xác với chữ viết tay.
	
	\end{itemize}
	
	
	
	\end{document}
	
	